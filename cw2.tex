\documentclass[a4paper]{article}
\usepackage[left=3.5cm, right=2.5cm, top=2.5cm, bottom=2.5cm]{geometry}
\usepackage[MeX]{polski}
\usepackage[utf8]{inputenc}
\usepackage{graphicx}
\usepackage{enumerate}
\usepackage{amsmath} %pakiet matematyczny
\usepackage{amssymb} %pakiet dodatkowych symboli
\title{Strona tytułowa}
\author{Lukasz Mrowka}
\date{Październik 2022}
\begin{document}
\maketitle
jakies informacje o mnie
\newpage
\textbf{paragraf 1}
Kto, naruszając, chociażby nieumyślnie, zasady bezpieczeństwa w ruchu lądowym, wodnym lub powietrznym, powoduje nieumyślnie wypadek, w którym inna osoba odniosła obrażenia ciała określone 
\newline
\begin{enumerate}
\item Punkt pierwszy spotkania
\item Punkt drugi spotkania
\end{enumerate}
\textbf{Naukowcy zidentyfikowali osobliwy ruch skręcający na orbitach dwóch zderzających się czarnych dziur, egzotyczne zjawisko przewidziane przez teorię grawitacji Einsteina.
Ich badania, opublikowane w Nature i prowadzone przez prof. Marka Hannama, dr. Charliego Hoya i dr. Jonathana Thompsona, donoszą, że po raz pierwszy efekt ten, znany jako precesja, został zaobserwowany w czarnych dziurach, gdzie skręcanie jest 10 miliardów razy szybsze niż w poprzednich obserwacjach.
Układ podwójny czarnych dziur został odkryty dzięki emisji przez nie fal grawitacyjnych na początku 2020 roku, przy użyciu detektorów LIGO i Virgo. Jedna z czarnych dziur, 40 razy masywniejsza od Słońca, jest prawdopodobnie najszybciej wirującą czarną dziurą, która została znaleziona dzięki falom grawitacyjnym. I w przeciwieństwie do wszystkich poprzednich obserwacji, szybko wirująca czarna dziura zniekształciła przestrzeń i czas tak bardzo, że cała orbita układu podwójnego kołysała się w przód i w tył.
Ta forma precesji jest charakterystyczna dla ogólnej teorii względności Einsteina. Wyniki te potwierdzają jej istnienie w najbardziej ekstremalnym zdarzeniu fizycznym, jakie możemy zaobserwować – zderzeniu dwóch czarnych dziur.
Bardziej przyziemnym przykładem precesji jest chwianie się wirującego bączka, który może chwiać się – lub precesować – raz na kilka sekund. Natomiast precesja w OTW jest zazwyczaj tak słabym efektem, że jest niezauważalna. W najszybszym przykładzie zmierzonym wcześniej na orbitujących gwiazdach neutronowych, zwanych pulsarami podwójnymi, precesja orbity zajęła ponad 75 lat. Podwójna czarna dziura w tym badaniu, potocznie znana jako GW200129 (nazwa pochodzi od daty jej zaobserwowania, 29 stycznia 2020 r.) precesuje kilka razy na sekundę – efekt 10 miliardów razy silniejszy niż zmierzony wcześniej.
Dr Jonathan Thompson wyjaśnił: Jest to bardzo trudny do zidentyfikowania efekt. Fale grawitacyjne są niezwykle słabe i ich wykrycie wymaga najczulszego aparatu pomiarowego w historii. Precesja jest jeszcze słabszym efektem ukrytym w i tak już słabym sygnale, więc musieliśmy przeprowadzić dokładną analizę, aby go odkryć.
 jeszcze dziwniejszy, niż sądzili.}
\end{document}


